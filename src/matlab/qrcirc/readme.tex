\documentclass{scrartcl}
\usepackage[T1]{fontenc}
\usepackage[utf8]{inputenc}
\usepackage{courier}

\usepackage{amsmath}
\usepackage{amssymb}
%\usepackage{amsfonts}

\usepackage[polish,english]{babel}
\usepackage{commath}
\usepackage{tgtermes}
\usepackage{newtxtext}
%%\usepackage{newtxmath} % FIXME: this breaks \dot + \mathrm
\usepackage{xifthen}
\usepackage[pdftex,colorlinks,allcolors=blue]{hyperref}
\usepackage{gnuplottex}

%%%%%%%%%%%%%%%%%%%%%%%%%%%%%%%%%%%%%%%%%%%%%%%%%%%%%%%%%%%%%%%%%%%%%%%%%%%%%%
%               Algebraic vectors and matrices
%%%%%%%%%%%%%%%%%%%%%%%%%%%%%%%%%%%%%%%%%%%%%%%%%%%%%%%%%%%%%%%%%%%%%%%%%%%%%%
%   \mMat{A}        Algebraic matrix
\newcommand\mMat[1]{\ensuremath{\boldsymbol{\mathrm{#1}}}}
%   \mVec{v}        Algebraic vector (single column algebraic matrix)
\newcommand\mVec[1]{\ensuremath{\boldsymbol{\mathrm{#1}}}}

%%%%%%%%%%%%%%%%%%%%%%%%%%%%%%%%%%%%%%%%%%%%%%%%%%%%%%%%%%%%%%%%%%%%%%%%%%%%%%
%               Enclosing arguments
%%%%%%%%%%%%%%%%%%%%%%%%%%%%%%%%%%%%%%%%%%%%%%%%%%%%%%%%%%%%%%%%%%%%%%%%%%%%%%
\newcommand\mOf[1]{\left({#1}\right)}

%%%%%%%%%%%%%%%%%%%%%%%%%%%%%%%%%%%%%%%%%%%%%%%%%%%%%%%%%%%%%%%%%%%%%%%%%%%%%%
%               Functionals, mappings, operators
%%%%%%%%%%%%%%%%%%%%%%%%%%%%%%%%%%%%%%%%%%%%%%%%%%%%%%%%%%%%%%%%%%%%%%%%%%%%%%
%   \mVnorm{X}      Vector norm
\newcommand\mVnorm[2][]{%
  \ifthenelse{\isempty{#1}}{\left\|{#2}\right\|}{\left\|{#2}\right\|_{#1}}}

\title{QRCirc}
\subtitle{QR factorization applied to DAEs describing point on circle (mathematical pendulum)}
\author{Paweł Tomulik}

\begin{document}

\maketitle

\begin{abstract}
This example explains some problems with integration of the Index-3 DAEs using
QR decomposition and standard explicit ODE solvers. We consider simple material
point whose motion is restricted to a unit circle $q_1^2 + q_2^2 = 1$. The
point gets non-zero initial velocity, and moves without friction, without
gravity and without external excitations (so it should preserve kinetic
energy). We demonstrate, that naive parametrization of the equations of motion
by direct use of analytical form of (continuous) QR factorization makes no
sense in some cases (such as the one being presented). After this, we reveal
numerical issues with integration methods based on QR factorization and
projections.
\end{abstract}

\section{Theory 0 - solving equations with QR decomposition}

An under-determined linear system $\mMat{A}\mVec{x} = \mVec{b}$ with $m \times
n$ matrix $\mMat{A}$, $m < n$ of full row rank, may be solved with QR
factorization. The standard approach is to decompose tall $n \times m$ matrix
$\mMat{A}^T$ as follows
%%%%%%%%%%%%%%%%%%%%%%%%%%%%%%%%%%%%%%%%%%%%%%%%%%%%%%%%%%%%%%%%%%%%%%%%%%%%%
\begin{equation}
  \mMat{A}^T
= \mMat{Q} \mMat{R}
= \begin{bmatrix} \mMat{Q}_1 & \mMat{Q}_2 \end{bmatrix}
  \cdot
  \begin{bmatrix} \mMat{R}_1 \\  \mMat{0} \end{bmatrix}
  \label{eq:INYH}
\end{equation}
%%%%%%%%%%%%%%%%%%%%%%%%%%%%%%%%%%%%%%%%%%%%%%%%%%%%%%%%%%%%%%%%%%%%%%%%%%%%%
where  $\mMat{Q}$ is $n \times n$ orthonormal, i.e. $\mMat{Q}^T \mMat{Q} =
\mMat{Q}\mMat{Q}^T = \mMat{I}_{m \times m}$, $\mMat{R}_1$ is $m \times m$ upper
triangular, invertible matrix. The columns of the $n \times m$ matrix
$\mMat{Q}_1$ form an orthonormal basis of the range space of $\mMat{A}$. The
columns of the $n \times (n-m)$ matrix $\mMat{Q}_2$ form an orthonormal basis
of the null-space of $\mMat{A}$.

The under-determined system \eqref{eq:INYH} has a solution that depends on
$(n-m) \times 1$ parameters $\mVec{z}$
%%%%%%%%%%%%%%%%%%%%%%%%%%%%%%%%%%%%%%%%%%%%%%%%%%%%%%%%%%%%%%%%%%%%%%%%%%%%%
\begin{equation}
  \mVec{x} = \mMat{S} \mVec{b} + \mMat{Q}_2 \mVec{z}
  \label{eq:1A2S}
\end{equation}
%%%%%%%%%%%%%%%%%%%%%%%%%%%%%%%%%%%%%%%%%%%%%%%%%%%%%%%%%%%%%%%%%%%%%%%%%%%%%
where $\mMat{S} = \mMat{Q}_1 \left( \mMat{R}_1^T \right)^{-1}$. Note that by
the orthogonality of $\mMat{Q}$ we have
%%%%%%%%%%%%%%%%%%%%%%%%%%%%%%%%%%%%%%%%%%%%%%%%%%%%%%%%%%%%%%%%%%%%%%%%%%%%%
\begin{equation}
  \mVec{z} = \mMat{Q}_2^T \mVec{x}
  \label{eq:36FN}
\end{equation}
%%%%%%%%%%%%%%%%%%%%%%%%%%%%%%%%%%%%%%%%%%%%%%%%%%%%%%%%%%%%%%%%%%%%%%%%%%%%%

%% %%%%%%%%%%%%%%%%%%%%%%%%%%%%%%%%%%%%%%%%%%%%%%%%%%%%%%%%%%%%%%%%%%%%%%%%%%%%%
%% \begin{equation}
%%   \mVec{x}
%% = \begin{bmatrix} \mMat{Q}_1 & \mMat{Q}_2 \end{bmatrix}
%%   \begin{bmatrix}
%%     \left(\mMat{R}_1^T\right)^{-1} \mVec{b}\\
%%     \mVec{z}
%%   \end{bmatrix}
%%   \label{eq:2PHV}
%% \end{equation}
%%%%%%%%%%%%%%%%%%%%%%%%%%%%%%%%%%%%%%%%%%%%%%%%%%%%%%%%%%%%%%%%%%%%%%%%%%%%%
%% with arbitrary $\mVec{z}$ (vector of size $d \times 1$).  As $\mVec{z}$ stands
%% for independent variable, i.e. for arbitrary values of $\mVec{z}$ the original
%% system of equations must be satisfied, it thus must hold
%% %%%%%%%%%%%%%%%%%%%%%%%%%%%%%%%%%%%%%%%%%%%%%%%%%%%%%%%%%%%%%%%%%%%%%%%%%%%%%
%% \begin{equation}
%%   \mMat{A} \mMat{Q}_2 = \mMat{0}_{m \times d}.
%% \end{equation}
%% %%%%%%%%%%%%%%%%%%%%%%%%%%%%%%%%%%%%%%%%%%%%%%%%%%%%%%%%%%%%%%%%%%%%%%%%%%%%%
%%
%% The QR factorization $\mMat{A}^T = \mMat{Q} \mMat{R}$ is unique if
%% $\mMat{A}$ has full rank and we require $\mMat{R}$ to have positive diagonal
%% elements ($\mMat{A} = \mMat{Q} \mMat{D} \cdot \mMat{D} \mMat{R}$ is a QR
%% factorization for any $\mMat{D} = \operatorname{diag}\mOf{\pm 1}$
%% \cite[p.~356]{higham:2002:accurracy:book}).  Choosing $\mMat{D} =
%% \operatorname{sign} \mOf{\operatorname{diag}{\mOf{\mMat{R}}}}$ we obtain the
%% unique factorization $\mMat{Q}\mMat{D} \cdot \mMat{D}\mMat{R}$ from any
%% valid decomposition $\mMat{Q} \mMat{R}$. The matrices $\mMat{Q}$ and
%% $\mMat{R}$ are not uniquely determined however, and for single matrix
%% $\mMat{A}$ there is many possible valid $\mMat{Q} \mMat{R}$ decompositions
%% (and algorithms are free to pick up any of them). The algorithm in Lapack, for
%% example, computes QR using Householder reflections and switches sign on
%% diagonal of $\mMat{R}$ to minimize roundoff errors, thus significant
%% discontinuities should be expected even if then one smoothly changes
%% $\mMat{A}$.
%%
%% \subsection{Continuity of QR decomposition}
%%
%% In our algorithm we need to have all coefficients continuous in time (because
%% we interpolate them with continuous functions). Mathematicians teach us some
%% unpleasant facts on functions $\mMat{Z}\mOf{\mMat{A}}$ representing basis
%% of the nullspace of $\mMat{A}$. First, $\mMat{Z}$ is not unique, second globally
%% continuous mappings $\mMat{Z}$ exist only in some particular cases, for example
%% 1-DOF case (nullspace of dimension one) -- Byrd, Schnabel '85. By ``globally''
%% we mean over all the $\mMat{A}$'s of full rank. Local continuity might be
%% achieved -- there is special recipe (call it $\mMat{Z}\mOf{\mMat{A}}$) for
%% a continuous nullspace basis of full-rank matrix $\mMat{A}$ being not so far
%% from some other full-rank matrix $\mMat{\hat A}$ for which we know $\mMat{\hat
%% Z}$ to be a nullspace basis. The maximum safe distance is then determined by
%% the smallest singular value $\sigma_n$ of $\mMat{\hat A}$. Note, that walking
%% from $\mMat{\hat Z}$ to $\mMat{Z}\mOf{\mMat{A}}$ one usually loses some crucial
%% properties present at $\mMat{\hat Z}$, for example the orthogonality is not
%% preserved (one may re-orthogonalize new nullspace basis by Gram-Schmidt, it
%% preserves continuity).
%%
%% \textbf{Question}: it is important question, if the unique $\mMat{Q}\mMat{R}$
%% decomposition with positive diagonal in $\mMat{R}$ done by Householder
%% reflections yields continuous $\mMat{Q}_2$? Note: this is not a question about
%% existence of such continuous mappings but we ask about particular algorithm
%% with particular class of inputs. The existence of such smooth mapping seems to
%% be true for $\mMat{A}\mOf{t}$ being a smooth function of \emph{single variable}
%% $t$ and having full rank everywhere along $t$. Remarks given by Byrd and
%% Schnabel pertain to functions $\mMat{A}\mOf{\mVec{x}}: \mSetReal^n \mapsto
%% \mSetReal^{n \times m}$, that is $\mMat{x} \in \mSetReal^n$ and are very
%% generic (thus seemingly pessimistic). On the other hand, we are interested in
%% case $\mMat{A}\mOf{t}$, $t \in \mSetReal$ ($t$ is one-dimensional). Maybe a
%% result of Ważewski (1935) could be helpful, but the paper is in French. It
%% says, that ``if $\mSetName{D}$ is homeomorphic to a sphere, and if
%% $\mMat{A}\mOf{\mVec{x}}$ has full rank for all $\mVec{x} \in \mSetName{D}$,
%% then the continuous function $\mMat{z}\mOf{\mMat{x}}$ determining nullspace
%% basis of $\mMat{A}\mOf{\mVec{x}}$ exists'' (anyone to interpret this?).
%%
%% Let $\mMat{A}^T\mOf{t} \in \mSetName{C}^k\mOf{\mSetReal, \mSetReal^{n\times m}},
%% n \ge m$ be univariate matrix function with full rank. Then there exist QR
%% decomposition of $\mMat{A}^T$ which is continuous w.r.t. $t$ everywhere the
%% $\mMat{A}\mOf{t}$ is. Moreover, the QR preservers smoothness, that is the
%% matrix functions $\mMat{Q}_1\mOf{t}$ and $\mMat{R}\mOf{t}$ are $\mSetName{C}^k$
%% and these matrices might be defined by ordinary differential equations that
%% involve $\mMat{A}\mOf{t}$ and initial values $\mMat{Q}\mOf{t_0}$,
%% $\mMat{R}\mOf{t_0}$ (see works of Dieci et al. and references therein).
%%
%% The unique ``economy size'' decomposition, i.e. the $\mMat{Q}_1 \mMat{R}_1$
%% with $\mMat{R}_1$ having positive diagonal is continuous as a function of
%% $\mVec{x} \in \mSetReal^n$ for $\mMat{A}\mOf{\mVec{x}}$, this is claim by
%% Coleman and Sorensen (1983). It's not clear if the continuity is local or
%% global. The same authors say, that the algorithm based on Householder
%% reflections may produce continuous nullspace basis, if feed properly (row
%% interchange)  or modified to produce this factorization.
%%
%% Perhaps answers to all these questions are well known, but I don't
%% know them for the moment. For that reason I'll rather avoid using nullspaces,
%% i.e. the QR based algorithm will be carried in $\mSetReal^n$, not in the
%% independent coordinates from $\mSetReal^{n-m}$.
%%
\section{Theory 1 - let's try to solve linear DAEs with QR}

It's tempting to try QR with linear DAEs. Consider the following linear
time-varying DAEs (that is, all coefficient matrices are functions of time $t$
and do not depend on coordinates and velocities)
%%%%%%%%%%%%%%%%%%%%%%%%%%%%%%%%%%%%%%%%%%%%%%%%%%%%%%%%%%%%%%%%%%%%%%%%%%%%%
\begin{subequations}
\label{eq:87B0}
\begin{align}
    \mMat{M} \mVec{\dot {\hat v}}
  + \mMat{\Phi}_{\mVec{q}}^T \mVec{\hat \lambda}
  & = - \mVec{F}_1  - \mMat{F}_{1,\mVec{v}} \mVec{\hat v}
                    - \mMat{F}_{1,\mVec{q}} \mVec{\hat q}
  \label{eq:G35J}
\\
  \mVec{\dot {\hat q}} & = \mVec{\hat v}
  \label{eq:U58A}
\\
  \mMat{\Phi}_{\mVec{q}} \mVec{\hat q} & = - \mVec{\Phi}
  \label{eq:4LWR}
\end{align}
\end{subequations}
%%%%%%%%%%%%%%%%%%%%%%%%%%%%%%%%%%%%%%%%%%%%%%%%%%%%%%%%%%%%%%%%%%%%%%%%%%%%%
which represent linearized Index-3 equations of motion. In \eqref{eq:87B0}, the
coefficients $\mMat{M}$, $\mMat{\Phi}_{\mVec{q}}$, $\mVec{F}_1$,
$\mMat{F}_{1,\mVec{v}}$, $\mMat{F}_{1,\mVec{q}}$ and  $\mVec{F}_2$, are assumed
to be known functions of scalar $t$, and the unknown functions are $\mVec{\hat
v}$, $\mVec{\hat q}$, $\mVec{\hat \lambda}$.

In what follows, we will try to determine the underlying ODEs for
\eqref{eq:87B0} using null-space method based on QR factorization of
$\mMat{\Phi}_{\mVec{q}}$. First, let's write-out explicitly all the constraint
equations for \eqref{eq:87B0}, including hidden constraints
%%%%%%%%%%%%%%%%%%%%%%%%%%%%%%%%%%%%%%%%%%%%%%%%%%%%%%%%%%%%%%%%%%%%%%%%%%%%%
\begin{subequations}
\label{eq:W1MV}
\begin{align}
  \mMat{\Phi}_{\mVec{q}} \mVec{\hat q}
& = - \mVec{\Phi}
\\
\mMat{\Phi}_{\mVec{q}} \mVec{\hat v}
& = - \mVec{\dot \Phi}
- \mMat{\dot \Phi}_{\mVec{q}} \mVec{\hat q}
    \equiv \mVec{\nu}
\\
\mMat{\Phi}_{\mVec{q}} \mVec{\dot {\hat v}}
& = - \mVec{\ddot \Phi}
- \mVec{\ddot \Phi}_{\mVec{q}} \mVec{\hat q}
    - 2 \, \mMat{\dot \Phi}_{\mVec{q}}
    \cdot \mVec{\hat v}
    \equiv \mVec{\gamma}
\end{align}
\end{subequations}
%%%%%%%%%%%%%%%%%%%%%%%%%%%%%%%%%%%%%%%%%%%%%%%%%%%%%%%%%%%%%%%%%%%%%%%%%%%%%

If $\mMat{\Phi}_{\mVec{q}}$ is continuous and has full row rank for all $t$ we
may try to naively assume that there exists continuous QR decomposition of
$\mMat{\Phi}_{\mVec{q}}^T$ (including the null-space basis $\mMat{Q}_2$). Then
by \eqref{eq:1A2S} we have the following solution to \eqref{eq:W1MV}
%%%%%%%%%%%%%%%%%%%%%%%%%%%%%%%%%%%%%%%%%%%%%%%%%%%%%%%%%%%%%%%%%%%%%%%%%%%%%
\begin{subequations}
\label{eq:KLREN}
\begin{align}
  \mVec{\hat q}
  & = - \mMat{S} \mVec{\Phi} + \mMat{Q}_2 \mVec{z}
  \label{eq:TMOIZ}
\\
  \mVec{\hat v}
  & = \mMat{S} \mVec{\nu} + \mMat{Q}_2 \mVec{s}
  \label{eq:J53RJ}
\\
  \mVec{\dot {\hat v}}
  & = \mMat{S} \mVec{\gamma} + \mMat{Q}_2 \mVec{a}
  \label{eq:XQKT1}
\end{align}
\end{subequations}
%%%%%%%%%%%%%%%%%%%%%%%%%%%%%%%%%%%%%%%%%%%%%%%%%%%%%%%%%%%%%%%%%%%%%%%%%%%%%
where $\mVec{z}$, $\mVec{s}$ and $\mVec{a}$ are
independent coordinates, velocities and accelerations respectively (unknown).

The independent accelerations may be obtained by substituting \eqref{eq:XQKT1}
in \eqref{eq:G35J} and left-multiplying \eqref{eq:G35J} by $\mMat{Q}_2^T$
%%%%%%%%%%%%%%%%%%%%%%%%%%%%%%%%%%%%%%%%%%%%%%%%%%%%%%%%%%%%%%%%%%%%%%%%%%%%%
\begin{equation}
  \mVec{a} = \left(\mMat{Q}_2^T \mMat{M} \mMat{Q}_2\right)^{-1}
  \mMat{Q}_2^T \cdot \left(
      \mVec{\hat g}
    - \mMat{M} \mMat{S} \mVec{\gamma}
  \right)
\end{equation}
%%%%%%%%%%%%%%%%%%%%%%%%%%%%%%%%%%%%%%%%%%%%%%%%%%%%%%%%%%%%%%%%%%%%%%%%%%%%%
where $\mVec{\hat g} = - \mVec{F}_1  - \mMat{F}_{1,\mVec{v}}
\mVec{\hat v} - \mMat{F}_{1,\mVec{q}} \mVec{\hat q}$.
Coordinates $\mVec{z}$ and velocities $\mVec{s}$ shall
generally be obtained from ODE integrator. To formulate these ODEs, we need
to express the derivatives $\mVec{\dot z}$ and $\mVec{\dot
s}$ as a functions of $\mVec{z}$ and $\mVec{s}$. By
\eqref{eq:36FN} we have
%%%%%%%%%%%%%%%%%%%%%%%%%%%%%%%%%%%%%%%%%%%%%%%%%%%%%%%%%%%%%%%%%%%%%%%%%%%%%
\begin{align}
  & \mVec{z} = \mMat{Q}_2^T \mVec{\hat q}, &
  & \mVec{s} = \mMat{Q}_2^T \mVec{\hat v} &
  \label{eq:71BI}
\end{align}
%%%%%%%%%%%%%%%%%%%%%%%%%%%%%%%%%%%%%%%%%%%%%%%%%%%%%%%%%%%%%%%%%%%%%%%%%%%%%
and thus
%%%%%%%%%%%%%%%%%%%%%%%%%%%%%%%%%%%%%%%%%%%%%%%%%%%%%%%%%%%%%%%%%%%%%%%%%%%%%
\begin{subequations}
\label{eq:S5B4P}
\begin{align}
  \mVec{\dot z}
  & = \mVec{s} + \mMat{\dot Q}_2^T \mVec{\hat q}
\label{eq:MNLPE}
\\
\mVec{\dot s}
& = \mVec{a} + \mMat{\dot Q}_2^T \mVec{\hat v}
\label{eq:JO0II}
\end{align}
\end{subequations}
%%%%%%%%%%%%%%%%%%%%%%%%%%%%%%%%%%%%%%%%%%%%%%%%%%%%%%%%%%%%%%%%%%%%%%%%%%%%%
where $\mVec{\hat q}$ shall be determined by \eqref{eq:TMOIZ} and
$\mVec{\hat v}$ by \eqref{eq:J53RJ}.

\subsection{What's wrong with this approach?}

We've assumed continuity of $\mVec{Q}_2$. A known fact is, that null-space
basis $\mVec{Q}_2$ is not uniquely determined in general. In consequence, its
continuity cannot be guaranteed, and state-of-art numerical procedures used to
generate discontinuous bases. This leads to three problems: (1) the definition
of independent coordinates $\mVec{z}$ changes at each discontinuity
point, (2) the derivative $\mMat{\dot Q}_2$ used in \eqref{eq:S5B4P} is
undefined at discontinuity points, (3) there is no way to predict where the
discontinuity points will appear.

\section{Theory 2 -- updated null-space basis}

Byrd and Schnabel cite after Goodman the following update formula for a locally
continuous null-space basis. If $\mMat{\bar Z}$ is any known null-space for
some matrix $\mMat{\bar A}$, and $\mMat{A}$ is not too far from $\mMat{\bar A}$
($\mVnorm{\mMat{A}-\mMat{\bar A}} < \sigma$, with $\sigma > 0$ being smallest
singular value of $\mMat{\bar A}$), then
%%%%%%%%%%%%%%%%%%%%%%%%%%%%%%%%%%%%%%%%%%%%%%%%%%%%%%%%%%%%%%%%%%%%%%%%%%%%%
\begin{equation}
  \mMat{z}\mOf{\mMat{A}} = \left[
    \mMat{I} - \mVec{\bar A}\left(\mMat{A}^T\mMat{\bar A}\right)^{-1}\mMat{A}^T
  \right] \mMat{\bar Z}
\end{equation}
%%%%%%%%%%%%%%%%%%%%%%%%%%%%%%%%%%%%%%%%%%%%%%%%%%%%%%%%%%%%%%%%%%%%%%%%%%%%%
is a null-space basis for $\mMat{A}$ (not necessarily orthogonal).
This recipe is used, for example, by Kim and Vanderploeg to reduce non-linear
equations of motion. If the null-space basis is determined by means of QR
factorization, the formula for the updated null-space basis $\mMat{\tilde Q}_2$
takes the following form
%%%%%%%%%%%%%%%%%%%%%%%%%%%%%%%%%%%%%%%%%%%%%%%%%%%%%%%%%%%%%%%%%%%%%%%%%%%%%
\begin{equation}
  \mMat{\tilde Q}_2 = \left[
    \mMat{I} - \mMat{\tilde S}\mMat{\Phi}_{\mVec{q}}
  \right] \mMat{\bar Q}_2
  \label{eq:O1FSO}
\end{equation}
%%%%%%%%%%%%%%%%%%%%%%%%%%%%%%%%%%%%%%%%%%%%%%%%%%%%%%%%%%%%%%%%%%%%%%%%%%%%%
where $\mMat{\bar Q}_1$ and $\mMat{\bar Q}_2$ represent the QR factors for some
known $\mMat{\bar \Phi}_{\mVec{q}} = \mMat{\Phi}_{\mVec{q}}\mOf{t_{*}}$, and
%%%%%%%%%%%%%%%%%%%%%%%%%%%%%%%%%%%%%%%%%%%%%%%%%%%%%%%%%%%%%%%%%%%%%%%%%%%%%
\begin{equation}
  \mMat{\tilde S} = \mMat{\bar Q}_1 \left(
    \mMat{\Phi}_{\mVec{q}} \mMat{\bar Q}_1
  \right)^{-1}
  \label{eq:H3998}
\end{equation}
%%%%%%%%%%%%%%%%%%%%%%%%%%%%%%%%%%%%%%%%%%%%%%%%%%%%%%%%%%%%%%%%%%%%%%%%%%%%%
Note, that $\mMat{\bar Q}_2^T \mMat{\tilde S} = \mMat{0}$,
$\mMat{\bar Q}_2^T \mMat{\tilde Q}_2 = \mMat{I}$,
$\mMat{\Phi}_{\mVec{q}} \mMat{\tilde S} = \mMat{I}$ and
$\mMat{\Phi}_{\mVec{q}} \mMat{\tilde Q}_2 = \mMat{0}$. The solutions
\eqref{eq:KLREN} to constraints equations now take the form
%%%%%%%%%%%%%%%%%%%%%%%%%%%%%%%%%%%%%%%%%%%%%%%%%%%%%%%%%%%%%%%%%%%%%%%%%%%%%
\begin{subequations}
\label{eq:LHUIY}
\begin{align}
  \mVec{\hat q}
  & = - \mMat{\tilde S} \mVec{\Phi} + \mMat{\tilde Q}_2 \mVec{z}
  \label{eq:NUYW4}
\\
  \mVec{\hat v}
  & = \mMat{\tilde S} \mVec{\nu} + \mMat{\tilde Q}_2 \mVec{s}
  \label{eq:FQVW3}
\\
  \mVec{\dot {\hat v}}
  & = \mMat{\tilde S} \mVec{\gamma} + \mMat{\tilde Q}_2 \mVec{a}
  \label{eq:JF01O}
\end{align}
\end{subequations}
%%%%%%%%%%%%%%%%%%%%%%%%%%%%%%%%%%%%%%%%%%%%%%%%%%%%%%%%%%%%%%%%%%%%%%%%%%%%%
The independent variables in \eqref{eq:LHUIY} are not same as in
\eqref{eq:KLREN}. They're determined as
%%%%%%%%%%%%%%%%%%%%%%%%%%%%%%%%%%%%%%%%%%%%%%%%%%%%%%%%%%%%%%%%%%%%%%%%%%%%%
\begin{align}
  & \mVec{z} = \mMat{\bar Q}_2^T \mVec{\hat q}, &
  & \mVec{s} = \mMat{\bar Q}_2^T \mVec{\hat v} &
  \label{eq:71BJ}
\end{align}
%%%%%%%%%%%%%%%%%%%%%%%%%%%%%%%%%%%%%%%%%%%%%%%%%%%%%%%%%%%%%%%%%%%%%%%%%%%%%
$\mMat{\bar Q}_2$ is constant within the neighborhood of $t_{*}$ being
considered, so $\mMat{\dot {\bar Q}}_2 = \mMat{0}$ and thus
%%%%%%%%%%%%%%%%%%%%%%%%%%%%%%%%%%%%%%%%%%%%%%%%%%%%%%%%%%%%%%%%%%%%%%%%%%%%%
\begin{subequations}
\begin{align}
  \mVec{\dot z} & = \mVec{s}
\\
\mVec{\dot s} & = \mVec{a}
\end{align}
\end{subequations}
%%%%%%%%%%%%%%%%%%%%%%%%%%%%%%%%%%%%%%%%%%%%%%%%%%%%%%%%%%%%%%%%%%%%%%%%%%%%%
where
%%%%%%%%%%%%%%%%%%%%%%%%%%%%%%%%%%%%%%%%%%%%%%%%%%%%%%%%%%%%%%%%%%%%%%%%%%%%%
\begin{equation}
  \mVec{a} = \left(\mMat{\tilde Q}_2^T \mMat{M} \mMat{\tilde Q}_2\right)^{-1}
  \mMat{\tilde Q}_2^T \cdot \left(
      \mVec{\hat g}
    - \mMat{M} \mMat{\tilde S} \mVec{\gamma}
  \right)
\end{equation}
%%%%%%%%%%%%%%%%%%%%%%%%%%%%%%%%%%%%%%%%%%%%%%%%%%%%%%%%%%%%%%%%%%%%%%%%%%%%%

With updated null-space basis we have more control on the discontinuity points.
The criterion for the recalculation of QR factors is given in Kim and
Vanderploeg.

\subsection{Integrating nonlinear DAEs as ODEs with QR factorization}

In what follows, we're solving index-3 equations of motions
%%%%%%%%%%%%%%%%%%%%%%%%%%%%%%%%%%%%%%%%%%%%%%%%%%%%%%%%%%%%%%%%%%%%%%%%%%%%%
\begin{subequations}
  \begin{align}
    \mMat{M} \mVec{\ddot q} + \mMat{\Phi}_{\mVec{q}}^T \mVec{\lambda}
      & = \mVec{g}\mOf{t,\mVec{q},\mVec{\dot q}}
    \\
    \mVec{\Phi}\mOf{t,\mVec{q}}
      &= \mVec{0}
  \end{align}
\end{subequations}
%%%%%%%%%%%%%%%%%%%%%%%%%%%%%%%%%%%%%%%%%%%%%%%%%%%%%%%%%%%%%%%%%%%%%%%%%%%%%

The presented approach is based on that of Kim and Vanderploeg. In the original
paper the authors were projecting the coordinates onto the constraints surface,
i.e. they were solving the following equations at certain time steps
%%%%%%%%%%%%%%%%%%%%%%%%%%%%%%%%%%%%%%%%%%%%%%%%%%%%%%%%%%%%%%%%%%%%%%%%%%%%%
\begin{equation}
  \begin{bmatrix}
    \mVec{\Phi} \mOf{\mVec{q}^{*}} \\
    \mMat{\bar Q}_2^T \left(\mVec{q}^{*} - \mVec{q} \right)
  \end{bmatrix}
  = \mVec{0}
  \label{eq:K1EVE}
\end{equation}
%%%%%%%%%%%%%%%%%%%%%%%%%%%%%%%%%%%%%%%%%%%%%%%%%%%%%%%%%%%%%%%%%%%%%%%%%%%%%
where $\mVec{q}^{*}$ denotes the coordinates projected on constraints surface.
The resultant Newton iteration was
%%%%%%%%%%%%%%%%%%%%%%%%%%%%%%%%%%%%%%%%%%%%%%%%%%%%%%%%%%%%%%%%%%%%%%%%%%%%%
\begin{subequations}
\label{eq:V5C00}
\begin{align}
  \label{eq:RI0GP}
  \begin{bmatrix}
    \mMat{\Phi}_{\mVec{q}}\mOf{\mVec{q}^{(k)}} \\
    \mMat{\bar Q}_2^T
  \end{bmatrix}
  \mVec{\Delta q}^{(k+1)}
  & = \begin{bmatrix}
    - \mVec{\Phi}\mOf{\mVec{q}^{(k)}} \\
    \mVec{0}
  \end{bmatrix}
  \\
  \label{eq:YMYOR}
  \mVec{q}^{(k+1)} & = \mVec{q}^{(k)} + \mVec{\Delta q}^{(k+1)}
\end{align}
\end{subequations}
%%%%%%%%%%%%%%%%%%%%%%%%%%%%%%%%%%%%%%%%%%%%%%%%%%%%%%%%%%%%%%%%%%%%%%%%%%%%%
The method required a specialized ODE integrator, where the "state" variables
were defined as follows
%%%%%%%%%%%%%%%%%%%%%%%%%%%%%%%%%%%%%%%%%%%%%%%%%%%%%%%%%%%%%%%%%%%%%%%%%%%%%
\begin{equation}
  \mVec{y} = \begin{bmatrix}
    \mVec{q} \\ \mVec{s}
  \end{bmatrix}
\end{equation}
%%%%%%%%%%%%%%%%%%%%%%%%%%%%%%%%%%%%%%%%%%%%%%%%%%%%%%%%%%%%%%%%%%%%%%%%%%%%%
with $\mVec{s}$ denoting a minimal set of velocities.

Note, that we can't apply traditional ODE integrator directly, because of the
projection \eqref{eq:K1EVE}. Taking explicit Euler as an example, the time step
would be
%%%%%%%%%%%%%%%%%%%%%%%%%%%%%%%%%%%%%%%%%%%%%%%%%%%%%%%%%%%%%%%%%%%%%%%%%%%%%
\begin{equation}
  \mVec{q}_{i+1} = \mVec{q}_i + h \cdot \mVec{\dot q}_i,
\end{equation}
%%%%%%%%%%%%%%%%%%%%%%%%%%%%%%%%%%%%%%%%%%%%%%%%%%%%%%%%%%%%%%%%%%%%%%%%%%%%%
whereas it should be
%%%%%%%%%%%%%%%%%%%%%%%%%%%%%%%%%%%%%%%%%%%%%%%%%%%%%%%%%%%%%%%%%%%%%%%%%%%%%
\begin{equation}
  \mVec{q}_{i+1} = \mVec{q}_i^{*} + h \cdot \mVec{\dot q}_i.
\end{equation}
%%%%%%%%%%%%%%%%%%%%%%%%%%%%%%%%%%%%%%%%%%%%%%%%%%%%%%%%%%%%%%%%%%%%%%%%%%%%%
Traditional solvers, however, have no interface to introduce the modified
values $\mVec{q}^{*}$ into the integration scheme.

\section{Example}

\subsection{The model}

The equation of motion are
%%%%%%%%%%%%%%%%%%%%%%%%%%%%%%%%%%%%%%%%%%%%%%%%%%%%%%%%%%%%%%%%%%%%%%%%%%%%%
\begin{subequations}
\label{eq:6HKMG}
\begin{align}
  \label{eq:X83SF}
  \mVec{\ddot q} + \mMat{\Phi}_{\mVec{q}}^T \mVec{\lambda} &= \mVec{0}
  \\
  \label{eq:HVFIS}
  \mVec{\Phi}\mOf{\mVec{q}} &= \mVec{0}
\end{align}
\end{subequations}
%%%%%%%%%%%%%%%%%%%%%%%%%%%%%%%%%%%%%%%%%%%%%%%%%%%%%%%%%%%%%%%%%%%%%%%%%%%%%

\subsection{Constraints and Jacobian}

The motion of the material point is restricted by the following constraint
%%%%%%%%%%%%%%%%%%%%%%%%%%%%%%%%%%%%%%%%%%%%%%%%%%%%%%%%%%%%%%%%%%%%%%%%%%%%%
\begin{equation}
  \mMat{\Phi}\mOf{\mVec{q}} \triangleq q_1^2 + q_2^2 - 1 = 0
  \label{eq:90ZU7}
\end{equation}
%%%%%%%%%%%%%%%%%%%%%%%%%%%%%%%%%%%%%%%%%%%%%%%%%%%%%%%%%%%%%%%%%%%%%%%%%%%%%
For this constraint the Jacobian matrix is
%%%%%%%%%%%%%%%%%%%%%%%%%%%%%%%%%%%%%%%%%%%%%%%%%%%%%%%%%%%%%%%%%%%%%%%%%%%%%
\begin{equation}
  \mMat{\Phi}_{\mVec{q}}\mOf{\mVec{q}} = 2 \cdot \begin{bmatrix}
    q_1 & q_2
  \end{bmatrix}
  \label{eq:5L0HL}
\end{equation}
%%%%%%%%%%%%%%%%%%%%%%%%%%%%%%%%%%%%%%%%%%%%%%%%%%%%%%%%%%%%%%%%%%%%%%%%%%%%%
The other invariants (hidden constraints) are
%%%%%%%%%%%%%%%%%%%%%%%%%%%%%%%%%%%%%%%%%%%%%%%%%%%%%%%%%%%%%%%%%%%%%%%%%%%%%
\begin{subequations}
\label{eq:IQT3I}
\begin{align}
  \mMat{\Phi}_{\mVec{q}}\mVec{\dot q} &= \mVec{0}
  \\
  \mMat{\Phi}_{\mVec{q}}\mVec{\ddot q} &= \mVec{\Gamma}
\end{align}
\end{subequations}
%%%%%%%%%%%%%%%%%%%%%%%%%%%%%%%%%%%%%%%%%%%%%%%%%%%%%%%%%%%%%%%%%%%%%%%%%%%%%
with
%%%%%%%%%%%%%%%%%%%%%%%%%%%%%%%%%%%%%%%%%%%%%%%%%%%%%%%%%%%%%%%%%%%%%%%%%%%%%
\begin{equation}
  \mVec{\Gamma} = - 2 \left( {\dot q}_2^2 + {\dot q}_2^2 \right)
\end{equation}
%%%%%%%%%%%%%%%%%%%%%%%%%%%%%%%%%%%%%%%%%%%%%%%%%%%%%%%%%%%%%%%%%%%%%%%%%%%%%

\subsection{QR decomposition}

QR decomposition $\mMat{\Phi}_{\mVec{q}}^T = \mMat{Q} \mMat{R}$ may be found in
an analytical form, and its factors are
%%%%%%%%%%%%%%%%%%%%%%%%%%%%%%%%%%%%%%%%%%%%%%%%%%%%%%%%%%%%%%%%%%%%%%%%%%%%%
\begin{subequations}
\begin{align}
  \mMat{Q}_1 & = \frac{1}{\sqrt{q_1^2 + q_1^2}} \begin{bmatrix}
    q_1 \\ q_2
  \end{bmatrix}
  \\
  \mMat{Q}_2 & = \frac{1}{\sqrt{q_1^2 + q_2^2}} \begin{bmatrix}
    q_2 \\ -q_1
  \end{bmatrix}
  \\
  \mMat{R}_1 & = 2 \sqrt{q_1^2 + q_2^2}
\end{align}
\end{subequations}
%%%%%%%%%%%%%%%%%%%%%%%%%%%%%%%%%%%%%%%%%%%%%%%%%%%%%%%%%%%%%%%%%%%%%%%%%%%%%

\subsection{Solving the hidden constraints (QR method)}

The hidden constraints \eqref{eq:IQT3I} are simply under-determined linear
equations, and their solutions are
%%%%%%%%%%%%%%%%%%%%%%%%%%%%%%%%%%%%%%%%%%%%%%%%%%%%%%%%%%%%%%%%%%%%%%%%%%%%%
\begin{align}
  \mVec{\dot q}   & = \mMat{Q}_2 \cdot s
  \label{eq:38DY7}
  \\
  \mVec{\ddot q}  & = \mMat{Q}_1 \mMat{R}_1^{-T} \mVec{\Gamma}
                    + \mMat{Q}_2 \cdot a
  \label{eq:EKLKN}
\end{align}
%%%%%%%%%%%%%%%%%%%%%%%%%%%%%%%%%%%%%%%%%%%%%%%%%%%%%%%%%%%%%%%%%%%%%%%%%%%%%
The physical interpretation for $s$ and $a$ is the following -- $s$ is the
modulus of point's velocity, whereas $\mMat{Q}_2 a$ is the tangential component
of its acceleration.

\subsection{Solving the nonlinear constraint (QR method)}

The positional constraint is nonlinear, and we have \textit{no general recipe}
for position parametrization. A naive decomposition $\mVec{q} = \mMat{Q}_1 r +
\mMat{Q}_2 z$ does not reveal anything useful, as we simply have $r \equiv 1$
and $z \equiv 0$ on each point of the unit circle (the orthogonal component $r$
is simply the circle radius).

For this particular case (that is for the point moving along circle) we could
propose the parameter $z$ to be length of the arc traversed by the point, so we
had $\mVec{q} = \begin{bmatrix} \cos{z}, & \sin{z} \end{bmatrix}^T$
and ${\dot z} = s$. Note, however, that this is simply an analytical solution
to the non-linear constraints~\eqref{eq:90ZU7} which is not available in
general case. Also, such a case-by-case parametrization has nothing to do with
the QR decomposition, so the work related to QR factorization is simply wasted.

Another approach, which is proposed by Kim and Vanderploeg, is to not
para\-me\-tri\-ze the positions for integration, and formulate the variables to
be integrated as $\mVec{y} = \begin{bmatrix} \mVec{q}^T, & \mVec{s}^T
\end{bmatrix}$. The authors then project the positions onto the constraints
surface orthogonally to some fixed hyperplane. We'll follow this approach, but
our projection will be orthogonal to the constraint surface.

Assuming, that $\mVec{q}^{(0)}$ is an inaccurate position provided by the
integrator (apart from constraints) we may search for the corrected position
$\mVec{q}$ by the following projection orthogonal to the constraints surface at
$\mVec{q}$
%%%%%%%%%%%%%%%%%%%%%%%%%%%%%%%%%%%%%%%%%%%%%%%%%%%%%%%%%%%%%%%%%%%%%%%%%%%%%
\begin{subequations}
\begin{align}
  (\mVec{q} - \mVec{q}^{(0)})^{T} (\mVec{q} - \mVec{q}^{(0)})
    & \rightarrow \min
  \\
  \text{w.r.t.}\;\;\mVec{\Phi}\mOf{\mVec{q}}
    & = \mVec{0}
\end{align}
\end{subequations}
%%%%%%%%%%%%%%%%%%%%%%%%%%%%%%%%%%%%%%%%%%%%%%%%%%%%%%%%%%%%%%%%%%%%%%%%%%%%%
This is equivalent to the following nonlinear problem
%%%%%%%%%%%%%%%%%%%%%%%%%%%%%%%%%%%%%%%%%%%%%%%%%%%%%%%%%%%%%%%%%%%%%%%%%%%%%
\begin{subequations}
\begin{align}
  \mVec{\Phi}\mOf{\mVec{q}} & = \mVec{0}
  \\
  \mMat{Q}_2^T\mOf{\mVec{q}}(\mVec{q} - \mVec{q}^{(0)}) &= \mVec{0}
\end{align}
\end{subequations}
%%%%%%%%%%%%%%%%%%%%%%%%%%%%%%%%%%%%%%%%%%%%%%%%%%%%%%%%%%%%%%%%%%%%%%%%%%%%%
which may be numerically solved by the following Newton iteration
%%%%%%%%%%%%%%%%%%%%%%%%%%%%%%%%%%%%%%%%%%%%%%%%%%%%%%%%%%%%%%%%%%%%%%%%%%%%%
\begin{subequations}
\begin{align}
  \begin{bmatrix}
    \mMat{\Phi}_{\mVec{q}}^{(k)} \\
    \mMat{J}_{\parallel}^{(k)}
  \end{bmatrix}
  \mVec{\Delta q}^{(k+1)}
  & = - \begin{bmatrix}
    \mVec{\Phi}^{(k)} \\
    \mMat{Q}_2^{T\,(k)} \left(\mVec{q}^{(k)} - \mVec{q}^{(0)}\right)
  \end{bmatrix}
  \\
  \mVec{q}^{(k+1)} &= \mVec{q}^{(k)} + \mVec{\Delta q}^{(k+1)}
\end{align}
\end{subequations}
%%%%%%%%%%%%%%%%%%%%%%%%%%%%%%%%%%%%%%%%%%%%%%%%%%%%%%%%%%%%%%%%%%%%%%%%%%%%%
where
%%%%%%%%%%%%%%%%%%%%%%%%%%%%%%%%%%%%%%%%%%%%%%%%%%%%%%%%%%%%%%%%%%%%%%%%%%%%%
\begin{equation}
  \mMat{J}_{\parallel}\mOf{\mVec{q}} = \frac{\partial}{\partial \mVec{q}}\left\{
    \mMat{Q}_2^{T}\mOf{\mVec{q}} \cdot \left(\mVec{q} - \mVec{q}^{(0)}\right)
  \right\}
\end{equation}
%%%%%%%%%%%%%%%%%%%%%%%%%%%%%%%%%%%%%%%%%%%%%%%%%%%%%%%%%%%%%%%%%%%%%%%%%%%%%
In our circle example we have
%%%%%%%%%%%%%%%%%%%%%%%%%%%%%%%%%%%%%%%%%%%%%%%%%%%%%%%%%%%%%%%%%%%%%%%%%%%%%
\begin{equation}
  \mMat{Q}_2^T \cdot \left(\mVec{q} - \mVec{q}^{(0)}\right)
  = \frac{q_1 q_2^{(0)} - q_2 q_1^{(0)}}{\sqrt{q_1^2 + q_2^2}}
  = \begin{bmatrix} q_2^{(0)}, & -q_1^{(0)} \end{bmatrix} \mMat{Q}_1
\end{equation}
%%%%%%%%%%%%%%%%%%%%%%%%%%%%%%%%%%%%%%%%%%%%%%%%%%%%%%%%%%%%%%%%%%%%%%%%%%%%%
and it may be shown, that
%%%%%%%%%%%%%%%%%%%%%%%%%%%%%%%%%%%%%%%%%%%%%%%%%%%%%%%%%%%%%%%%%%%%%%%%%%%%%
\begin{equation}
  \frac{\dif}{\dif t}\left\{
    \mMat{Q}_2^T \cdot \left(\mVec{q} - \mVec{q}^{(0)}\right)
  \right\}
  = \frac{\begin{bmatrix} q_2^{(0)}, & -q_1^{(0)} \end{bmatrix}}{\mVnorm{\mVec{q}}}
    \left(\mMat{I} + \mMat{Q}_1\mMat{Q}_1^T\right) \mVec{\dot q}
  = \mMat{J}_{\parallel} \mVec{\dot q}
\end{equation}
%%%%%%%%%%%%%%%%%%%%%%%%%%%%%%%%%%%%%%%%%%%%%%%%%%%%%%%%%%%%%%%%%%%%%%%%%%%%%
thus
%%%%%%%%%%%%%%%%%%%%%%%%%%%%%%%%%%%%%%%%%%%%%%%%%%%%%%%%%%%%%%%%%%%%%%%%%%%%%
\begin{equation}
  \mMat{J}_{\parallel}
= \frac{\begin{bmatrix} q_2^{(0)}, & -q_1^{(0)} \end{bmatrix}}{\mVnorm{\mVec{q}}}
  \left(\mMat{I} + \mMat{Q}_1\mMat{Q}_1^T\right)
\end{equation}

\subsection{Determining the ODEs (QR method)}

Our RHS function will compute the following vector of derivatives
%%%%%%%%%%%%%%%%%%%%%%%%%%%%%%%%%%%%%%%%%%%%%%%%%%%%%%%%%%%%%%%%%%%%%%%%%%%%%
\begin{equation}
  \mVec{\dot y} = \begin{bmatrix}
    \mVec{\dot q} \\
    \mVec{\dot s}
  \end{bmatrix}
\end{equation}
%%%%%%%%%%%%%%%%%%%%%%%%%%%%%%%%%%%%%%%%%%%%%%%%%%%%%%%%%%%%%%%%%%%%%%%%%%%%%
The $\mVec{\dot q}$ is already defined by \eqref{eq:38DY7}, and for ${\dot s}$
we use fact that $s = \mMat{Q}_2^T \mVec{\dot q}$, so
%%%%%%%%%%%%%%%%%%%%%%%%%%%%%%%%%%%%%%%%%%%%%%%%%%%%%%%%%%%%%%%%%%%%%%%%%%%%%
\begin{equation}
  {\dot s} =  \frac{\dif}{\dif t}\left\{\mMat{Q}_2^T\mVec{\dot q}\right\}
  = \mMat{\dot Q}_2^T \mVec{\dot q}
  + \mMat{Q}_2^T \mVec{\ddot q}
  = \mMat{\dot Q}_2^T \mVec{Q}_2 s
  + \mMat{Q}_2^T \mMat{Q}_1 \mMat{R}_1^{-T} \mVec{\Gamma}
  + \mMat{Q}_2^T \mMat{Q}_2 a
  = a
\end{equation}
%%%%%%%%%%%%%%%%%%%%%%%%%%%%%%%%%%%%%%%%%%%%%%%%%%%%%%%%%%%%%%%%%%%%%%%%%%%%%
because $\mMat{\dot Q}_2^T\mMat{Q}_2 = 0$ \footnote{Note, that
$\mMat{\dot Q}_2^T\mMat{Q}_2 = - \mMat{Q}_2^T \mMat{\dot Q}_2$ (because
$\mMat{Q}_2^T\mMat{Q}_2 = 1 = const)$. But in our (1-DOF) case it's it must
hold $\mMat{\dot Q}_2^T\mMat{Q}_2 = \mMat{Q}_2^T\mMat{\dot Q}_2$ at the same
time, so the product may only be zero.}.
%%%%%%%%%%%%%%%%%%%%%%%%%%%%%%%%%%%%%%%%%%%%%%%%%%%%%%%%%%%%%%%%%%%%%%%%%%%%%
The variable $a$ might be determined from the equation of motion
\eqref{eq:X83SF} and \eqref{eq:EKLKN} from which we simply get
%%%%%%%%%%%%%%%%%%%%%%%%%%%%%%%%%%%%%%%%%%%%%%%%%%%%%%%%%%%%%%%%%%%%%%%%%%%%%
\begin{equation}
  a = 0
\end{equation}
%%%%%%%%%%%%%%%%%%%%%%%%%%%%%%%%%%%%%%%%%%%%%%%%%%%%%%%%%%%%%%%%%%%%%%%%%%%%%
Finally, our ODEs may be written as
%%%%%%%%%%%%%%%%%%%%%%%%%%%%%%%%%%%%%%%%%%%%%%%%%%%%%%%%%%%%%%%%%%%%%%%%%%%%%
\begin{subequations}
  \begin{align}
    \mVec{\dot q} &= \mMat{Q}_2 s
    \\
    {\dot s} &= 0
  \end{align}
\end{subequations}
%%%%%%%%%%%%%%%%%%%%%%%%%%%%%%%%%%%%%%%%%%%%%%%%%%%%%%%%%%%%%%%%%%%%%%%%%%%%%

\section{Experiment results}

In our experiments we run the integration for 1000 seconds and plot the
trajectory. We start from point $\mVec{q}_0 = \begin{bmatrix}1, &
0\end{bmatrix}^T$ and velocity $s = 1$. In an ideal situation the point should
move on the unit circle centered at zero.

\subsection{Direct use of continuous QR factorization}

In this section we'll present results of two experiments. In the first
experiment the equations of motion are integrated using ODE45 with QR
factorization and ``internal'' projections. By ``internal'' projections we
mean, that the projections are made only internally within the RHS procedure to
compute $\mMat{Q}_2(q)$, but the integrator still uses state $q^{(0)}$ which is
not projected. The resultant trajectory is plotted on
figure~\ref{fig:QEQGJ}.
%%%%%%%%%%%%%%%%%%%%%%%%%%%%%%%%%%%%%%%%%%%%%%%%%%%%%%%%%%%%%%%%%%%%%%%%%%%%%
\begin{figure}[htbp]
  \begin{center}
    \begin{gnuplot}[terminal=epslatex,terminaloptions=color]
      set size ratio 1;
      set grid;
      set key left top;
      set style line 1 default;
      set title "ode45 + QR with (int) projections";
      #set ylabel "$_{(1)}$";
      set xlabel "$q_1$";
      set ylabel "$q_2$";
      plot "qrcirc1.csv" using 1:2 title "$q_1,q_2$" with lines ls 1;
    \end{gnuplot}
    \caption{Trajectory of the point for ODE45 with QR and internal projection}
    \label{fig:QEQGJ}
  \end{center}
\end{figure}
%%%%%%%%%%%%%%%%%%%%%%%%%%%%%%%%%%%%%%%%%%%%%%%%%%%%%%%%%%%%%%%%%%%%%%%%%%%%%

In the second experiment the equations of motion are integrated using ODE45 with
QR factorization and ``internal'' + ``external'' projections. We simply perform
few (up to five) integration steps and stop integration, then we project
position at the end and start the integration (for the next few steps) starting
from the updated position. The trajectory is shown on figure \ref{fig:S4ETL}.
%%%%%%%%%%%%%%%%%%%%%%%%%%%%%%%%%%%%%%%%%%%%%%%%%%%%%%%%%%%%%%%%%%%%%%%%%%%%%
\begin{figure}[htbp]
  \begin{center}
    \begin{gnuplot}[terminal=epslatex,terminaloptions=color]
      set size ratio 1;
      set grid;
      set key left top;
      set style line 1 default;
      set title "ode45 + QR with (int/ext) projections";
      #set ylabel "$_{(1)}$";
      set xlabel "$q_1$";
      set ylabel "$q_2$";
      set xrange [-1.5:1.5];
      set yrange [-1.5:1.5];
      plot "qrcirc2.csv" using 1:2 title "$q_1,q_2$" with lines ls 1;
    \end{gnuplot}
    \caption{Trajectory of the point for ODE45 with QR and internal + external projection}
    \label{fig:S4ETL}
  \end{center}
\end{figure}
%%%%%%%%%%%%%%%%%%%%%%%%%%%%%%%%%%%%%%%%%%%%%%%%%%%%%%%%%%%%%%%%%%%%%%%%%%%%%

\subsection{QR-update method}

The ``QR-update'' method is the one proposed by Kim and Vanderploeg. The method
uses fixed coordinate system $\mMat{\bar Q}_1 = \mMat{Q}_1\mOf{\mVec{\bar q}}$,
$\mMat{\bar Q}_2 = \mVec{Q}_2\mOf{\mVec{\bar q}}$ generated at some point
$\mVec{\bar q}$ satisfying constraints and an updated coordinate system
$\mVec{\tilde Q}_1\mOf{\mVec{q}}$, $\mVec{\tilde Q}_2\mOf{\mVec{q}}$ at
$\mVec{q}$ in a neighborhood of $\mVec{\bar q}$.


In what follows, we'll express the solution to nonlinear constraints as
%%%%%%%%%%%%%%%%%%%%%%%%%%%%%%%%%%%%%%%%%%%%%%%%%%%%%%%%%%%%%%%%%%%%%%%%%%%%%
\begin{equation}
  \mVec{q} = \mMat{\bar Q}_1 w + \mMat{\bar Q}_2 z
\end{equation}
%%%%%%%%%%%%%%%%%%%%%%%%%%%%%%%%%%%%%%%%%%%%%%%%%%%%%%%%%%%%%%%%%%%%%%%%%%%%%

\subsection{Solving the hidden constraints (QR-update method)}

The hidden constraints \eqref{eq:IQT3I} are solved as
%%%%%%%%%%%%%%%%%%%%%%%%%%%%%%%%%%%%%%%%%%%%%%%%%%%%%%%%%%%%%%%%%%%%%%%%%%%%%
\begin{align}
  \mVec{\dot q} &= \mMat{\tilde Q}_2 {\dot z}
  \\
  \mVec{\ddot q} &= \mMat{\tilde Q}_2 {\ddot z} + \mMat{\tilde S} \Gamma
\end{align}
%%%%%%%%%%%%%%%%%%%%%%%%%%%%%%%%%%%%%%%%%%%%%%%%%%%%%%%%%%%%%%%%%%%%%%%%%%%%%
where $\mMat{\tilde Q}_2$ and $\mMat{\tilde S}$ are given by \eqref{eq:O1FSO}
and \eqref{eq:H3998} respectively.

\subsection{Solving the nonlinear constraint (QR-update method)}

To solve the non-linear constraints we formulate the following problem
%%%%%%%%%%%%%%%%%%%%%%%%%%%%%%%%%%%%%%%%%%%%%%%%%%%%%%%%%%%%%%%%%%%%%%%%%%%%%
\begin{subequations}
\begin{align}
  \mMat{\Phi}\mOf{\mVec{q}} &= 0
  \\
  \mMat{\bar Q}_2^T \left(\mVec{q} - \mVec{\bar q}\right) &= z
\end{align}
\end{subequations}
%%%%%%%%%%%%%%%%%%%%%%%%%%%%%%%%%%%%%%%%%%%%%%%%%%%%%%%%%%%%%%%%%%%%%%%%%%%%%
this is solved by the following Newton iteration
%%%%%%%%%%%%%%%%%%%%%%%%%%%%%%%%%%%%%%%%%%%%%%%%%%%%%%%%%%%%%%%%%%%%%%%%%%%%%
\begin{subequations}
\begin{align}
  \begin{bmatrix}
    \mMat{\Phi}_{\mVec{q}}^{(k)} \\
    \mMat{\bar Q}_2^T
  \end{bmatrix}
  \mVec{\Delta q}^{(k+1)}
& = - \begin{bmatrix}
  \mVec{\Phi}^{(k)} \\
  0
\end{bmatrix}
\\
\mVec{q}^{(k+1)}
&= \mVec{q}^{(k)} + \mVec{\Delta q}^{(k+1)}
\end{align}
\end{subequations}
%%%%%%%%%%%%%%%%%%%%%%%%%%%%%%%%%%%%%%%%%%%%%%%%%%%%%%%%%%%%%%%%%%%%%%%%%%%%%
with $\mVec{q}^{(0)} = \mVec{\bar q} + \mMat{\bar Q}_2 z$.

\subsection{Determining the ODEs (QR-update method)}

The independent accelerations ${\ddot z}$ are determined as
%%%%%%%%%%%%%%%%%%%%%%%%%%%%%%%%%%%%%%%%%%%%%%%%%%%%%%%%%%%%%%%%%%%%%%%%%%%%%
\begin{equation}
  {\ddot z} = - \left(\mMat{\tilde Q}_2^T\mMat{\tilde Q}_2\right)^{-1}
                \mMat{\tilde Q}_2^T \mMat{\tilde S} \Gamma
\end{equation}
%%%%%%%%%%%%%%%%%%%%%%%%%%%%%%%%%%%%%%%%%%%%%%%%%%%%%%%%%%%%%%%%%%%%%%%%%%%%%
and the resultant ODEs may be formulated by defining the following independent
variables
%%%%%%%%%%%%%%%%%%%%%%%%%%%%%%%%%%%%%%%%%%%%%%%%%%%%%%%%%%%%%%%%%%%%%%%%%%%%%
\begin{equation}
  \mVec{y} = \begin{bmatrix} z \\ {\dot z} \end{bmatrix}
\end{equation}
%%%%%%%%%%%%%%%%%%%%%%%%%%%%%%%%%%%%%%%%%%%%%%%%%%%%%%%%%%%%%%%%%%%%%%%%%%%%%
and then
%%%%%%%%%%%%%%%%%%%%%%%%%%%%%%%%%%%%%%%%%%%%%%%%%%%%%%%%%%%%%%%%%%%%%%%%%%%%%
\begin{equation}
  \mVec{\dot y} = \begin{bmatrix}
    y_2
    \\
    - \left(\mMat{\tilde Q}_2^T\mMat{\tilde Q}_2\right)^{-1}
      \mMat{\tilde Q}_2^T \mMat{\tilde S} \Gamma
  \end{bmatrix}
\end{equation}
%%%%%%%%%%%%%%%%%%%%%%%%%%%%%%%%%%%%%%%%%%%%%%%%%%%%%%%%%%%%%%%%%%%%%%%%%%%%%

\subsection{Results (QR-update methods)}

The trajectory from QR-update method is presented in figure \ref{fig:C1W1M}.
The error in the kinetic energy is presented on figures \ref{fig:VV69A} and
\ref{fig:5MXRG} for default ODE45 setting and for tolerances set to $10^{-9}$
respectively.

%%%%%%%%%%%%%%%%%%%%%%%%%%%%%%%%%%%%%%%%%%%%%%%%%%%%%%%%%%%%%%%%%%%%%%%%%%%%%
\begin{figure}[htbp]
  \begin{center}
    \begin{gnuplot}[terminal=epslatex,terminaloptions=color]
      set size ratio 1;
      set grid;
      set key left top;
      set style line 1 default;
      set title "ode45 + QR-update";
      #set ylabel "$_{(1)}$";
      set xlabel "$q_1$";
      set ylabel "$q_2$";
      set xrange [-1.5:1.5];
      set yrange [-1.5:1.5];
      plot "qrcirc3q.csv" using 1:2 title "$q_1,q_2$" with lines ls 1;
    \end{gnuplot}
    \caption{Trajectory of the point for ODE45 with QR-update}
    \label{fig:C1W1M}
  \end{center}
\end{figure}
%%%%%%%%%%%%%%%%%%%%%%%%%%%%%%%%%%%%%%%%%%%%%%%%%%%%%%%%%%%%%%%%%%%%%%%%%%%%%

%%%%%%%%%%%%%%%%%%%%%%%%%%%%%%%%%%%%%%%%%%%%%%%%%%%%%%%%%%%%%%%%%%%%%%%%%%%%%
\begin{figure}[htbp]
  \begin{center}
    \begin{gnuplot}[terminal=epslatex,terminaloptions=color]
      set size ratio 1;
      set grid;
      set key left top;
      set style line 1 default;
      set title "ode45 with default tolerances + QR-update";
      #set ylabel "$_{(1)}$";
      set xlabel "$t$";
      set ylabel "$E_k$";
      set xrange [0:1000];
      set yrange [0:3.5];
      plot "qrcirc3e.csv" using 1:2 title "$(E_k - 0.5)$" with lines ls 1;
    \end{gnuplot}
    \caption{Error in kinetic energy of the point for ODE45 with default
             tolerances and QR-update}
    \label{fig:VV69A}
  \end{center}
\end{figure}
%%%%%%%%%%%%%%%%%%%%%%%%%%%%%%%%%%%%%%%%%%%%%%%%%%%%%%%%%%%%%%%%%%%%%%%%%%%%%

%%%%%%%%%%%%%%%%%%%%%%%%%%%%%%%%%%%%%%%%%%%%%%%%%%%%%%%%%%%%%%%%%%%%%%%%%%%%%
\begin{figure}[htbp]
  \begin{center}
    \begin{gnuplot}[terminal=epslatex,terminaloptions=color]
      set size ratio 1;
      set grid;
      set key left top;
      set style line 1 default;
      set title "ode45 (with tolerances set to 1E-9) + QR-update";
      #set ylabel "$_{(1)}$";
      set xlabel "$t$";
      set ylabel "$E_k$";
      set xrange [0:1000];
      set yrange [0:8e-7];
      plot "qrcirc3e2.csv" using 1:2 title "$(E_k - 0.5)$" with lines ls 1;
    \end{gnuplot}
    \caption{Error in kinetic energy of the point for ODE45 with tolerances set
             to 1E-9 and QR-update}
    \label{fig:5MXRG}
  \end{center}
\end{figure}
%%%%%%%%%%%%%%%%%%%%%%%%%%%%%%%%%%%%%%%%%%%%%%%%%%%%%%%%%%%%%%%%%%%%%%%%%%%%%


% P. Tomulik: my bibliography
%%\bibliographystyle{spmpsci}
%%\bibliography{readme}
% P. Tomulik: end

\end{document}
% end of file template.tex

% \label{?:45K37}
% \label{?:GN2B2}
% \label{?:TCRQA}
% \label{?:X9Z2K}
% \label{?:2A6JG}
% \label{?:WBSNK}

% vim: set expandtab syntax=tex tabstop=2 shiftwidth=2:
% vim: set foldmethod=marker foldcolumn=4:

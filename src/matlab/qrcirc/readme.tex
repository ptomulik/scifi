\documentclass{scrartcl}
\usepackage[T1]{fontenc}
\usepackage[utf8]{inputenc}
\usepackage{amsmath}
\usepackage{amssymb}
\usepackage[polish,english]{babel}
\usepackage{commath}
\usepackage{xifthen}
\usepackage{courier}
\usepackage{tgtermes,newtxtext,newtxmath}
%

%%%%%%%%%%%%%%%%%%%%%%%%%%%%%%%%%%%%%%%%%%%%%%%%%%%%%%%%%%%%%%%%%%%%%%%%%%%%%%
%               Algebraic vectors and matrices
%%%%%%%%%%%%%%%%%%%%%%%%%%%%%%%%%%%%%%%%%%%%%%%%%%%%%%%%%%%%%%%%%%%%%%%%%%%%%%
%   \mMat{A}        Algebraic matrix
\newcommand\mMat[1]{\ensuremath{\boldsymbol{\mathrm{#1}}}}
%   \mVec{v}        Algebraic vector (single column algebraic matrix)
\newcommand\mVec[1]{\ensuremath{\boldsymbol{\mathrm{#1}}}}

%%%%%%%%%%%%%%%%%%%%%%%%%%%%%%%%%%%%%%%%%%%%%%%%%%%%%%%%%%%%%%%%%%%%%%%%%%%%%%
%               Enclosing arguments
%%%%%%%%%%%%%%%%%%%%%%%%%%%%%%%%%%%%%%%%%%%%%%%%%%%%%%%%%%%%%%%%%%%%%%%%%%%%%%
\newcommand\mOf[1]{\left({#1}\right)}

%%%%%%%%%%%%%%%%%%%%%%%%%%%%%%%%%%%%%%%%%%%%%%%%%%%%%%%%%%%%%%%%%%%%%%%%%%%%%%
%               Functionals, mappings, operators
%%%%%%%%%%%%%%%%%%%%%%%%%%%%%%%%%%%%%%%%%%%%%%%%%%%%%%%%%%%%%%%%%%%%%%%%%%%%%%
%   \mVnorm{X}      Vector norm
\newcommand\mVnorm[2][]{%
  \ifthenelse{\isempty{#1}}{\left\|{#2}\right\|}{\left\|{#2}\right\|_{#1}}}

\title{QRCirc}
\subtitle{QR factorization applied to DAEs describing point on circle (mathematical pendulum)}
\author{Paweł Tomulik}

\begin{document}

\maketitle

\begin{abstract}
This example explains some problems with integration of the Index-3 DAEs using
QR decomposition and standard explicit ODE solvers. We consider simple material
point whose motion is restricted to a unit circle $q_1^2 + q_2^2 = 1$. The
point gets non-zero initial velocity, and moves without friction, without
gravity and without external excitations (so it should preserve kinetic
energy). We demonstrate, that naive parametrization of the equations of motion
by direct use of analytical form of (continuous) QR factorization makes no
sense in some cases (like the presented one). After this, we reveal numerical
issues with integration methods based on QR factorization and projections.
\end{abstract}

\section{The model}
The equation of motion are
%%%%%%%%%%%%%%%%%%%%%%%%%%%%%%%%%%%%%%%%%%%%%%%%%%%%%%%%%%%%%%%%%%%%%%%%%%%%%
\begin{subequations}
\label{eq:6HKMG}
\begin{align}
  \label{eq:X83SF}
  \mVec{\ddot q} + \mMat{\Phi}_{\mVec{q}}^T \mVec{\lambda} &= \mVec{0}
  \\
  \label{eq:HVFIS}
  \mVec{\Phi}\mOf{\mVec{q}} &= \mVec{0}
\end{align}
\end{subequations}
%%%%%%%%%%%%%%%%%%%%%%%%%%%%%%%%%%%%%%%%%%%%%%%%%%%%%%%%%%%%%%%%%%%%%%%%%%%%%

\section{Constraints and Jacobian}

The motion of the material point is restricted by the following constraint
%%%%%%%%%%%%%%%%%%%%%%%%%%%%%%%%%%%%%%%%%%%%%%%%%%%%%%%%%%%%%%%%%%%%%%%%%%%%%
\begin{equation}
  \mMat{\Phi}\mOf{\mVec{q}} \triangleq q_1^2 + q_2^2 - 1 = 0
  \label{eq:90ZU7}
\end{equation}
%%%%%%%%%%%%%%%%%%%%%%%%%%%%%%%%%%%%%%%%%%%%%%%%%%%%%%%%%%%%%%%%%%%%%%%%%%%%%
For this constraint the Jacobian matrix is
%%%%%%%%%%%%%%%%%%%%%%%%%%%%%%%%%%%%%%%%%%%%%%%%%%%%%%%%%%%%%%%%%%%%%%%%%%%%%
\begin{equation}
  \mMat{\Phi}_{\mVec{q}}\mOf{\mVec{q}} = 2 \cdot \begin{bmatrix}
    q_1 & q_2
  \end{bmatrix}
  \label{eq:5L0HL}
\end{equation}
%%%%%%%%%%%%%%%%%%%%%%%%%%%%%%%%%%%%%%%%%%%%%%%%%%%%%%%%%%%%%%%%%%%%%%%%%%%%%
The other invariants (hidden constraints) are
%%%%%%%%%%%%%%%%%%%%%%%%%%%%%%%%%%%%%%%%%%%%%%%%%%%%%%%%%%%%%%%%%%%%%%%%%%%%%
\begin{subequations}
\label{eq:IQT3I}
\begin{align}
  \mMat{\Phi}_{\mVec{q}}\mVec{\dot q} &= \mVec{0}
  \\
  \mMat{\Phi}_{\mVec{q}}\mVec{\ddot q} &= \mVec{\Gamma}
\end{align}
\end{subequations}
%%%%%%%%%%%%%%%%%%%%%%%%%%%%%%%%%%%%%%%%%%%%%%%%%%%%%%%%%%%%%%%%%%%%%%%%%%%%%
with
%%%%%%%%%%%%%%%%%%%%%%%%%%%%%%%%%%%%%%%%%%%%%%%%%%%%%%%%%%%%%%%%%%%%%%%%%%%%%
\begin{equation}
  \mVec{\Gamma} = - 2 \left( {\dot q}_2^2 + {\dot q}_2^2 \right)
\end{equation}
%%%%%%%%%%%%%%%%%%%%%%%%%%%%%%%%%%%%%%%%%%%%%%%%%%%%%%%%%%%%%%%%%%%%%%%%%%%%%

\section{QR decomposition}

QR decomposition $\mMat{\Phi}_{\mVec{q}}^T = \mMat{Q} \mMat{R}$ may be found in
an analytical form, and its factors are
%%%%%%%%%%%%%%%%%%%%%%%%%%%%%%%%%%%%%%%%%%%%%%%%%%%%%%%%%%%%%%%%%%%%%%%%%%%%%
\begin{subequations}
\begin{align}
  \mMat{Q}_1 & = \frac{1}{\sqrt{q_1^2 + q_1^2}} \begin{bmatrix}
    q_1 \\ q_2
  \end{bmatrix}
  \\
  \mMat{Q}_2 & = \frac{1}{\sqrt{q_1^2 + q_2^2}} \begin{bmatrix}
    q_2 \\ -q_1
  \end{bmatrix}
  \\
  \mMat{R}_1 & = 2 \sqrt{q_1^2 + q_2^2}
\end{align}
\end{subequations}
%%%%%%%%%%%%%%%%%%%%%%%%%%%%%%%%%%%%%%%%%%%%%%%%%%%%%%%%%%%%%%%%%%%%%%%%%%%%%

\section{Solving the hidden constraints (QR method)}

The hidden constraints \eqref{eq:IQT3I} are simply under-determined linear
equations, and their solutions are
%%%%%%%%%%%%%%%%%%%%%%%%%%%%%%%%%%%%%%%%%%%%%%%%%%%%%%%%%%%%%%%%%%%%%%%%%%%%%
\begin{align}
  \mVec{\dot q}   & = \mMat{Q}_2 \cdot s
  \label{eq:38DY7}
  \\
  \mVec{\ddot q}  & = \mMat{Q}_1 \mMat{R}_1^{-T} \mVec{\Gamma}
                    + \mMat{Q}_2 \cdot a
  \label{eq:EKLKN}
\end{align}
%%%%%%%%%%%%%%%%%%%%%%%%%%%%%%%%%%%%%%%%%%%%%%%%%%%%%%%%%%%%%%%%%%%%%%%%%%%%%
The physical interpretation for $s$ and $a$ is the following -- $s$ is the
modulus of point's velocity, whereas $\mMat{Q}_2 a$ is the tangential component
of its acceleration.

\section{Solving the nonlinear constraint (QR method)}

The positional constraint is nonlinear, and we have \textit{no general recipe}
for position parametrization. A naive decomposition $\mVec{q} = \mMat{Q}_1 r +
\mMat{Q}_2 z$ does not reveal anything useful, as we simply have $r \equiv 1$
and $z \equiv 0$ on each point of the unit circle (the orthogonal component $r$
is simply the circle radius).

For this particular case (that is for the point moving along circle) we could
propose the parameter $z$ to be length of the arc traversed by the point, so we
had $\mVec{q} = \begin{bmatrix} \cos{z}, & \sin{z} \end{bmatrix}^T$
and ${\dot z} = s$. Note, however, that this is simply an analytical solution
to the non-linear constraints~\eqref{eq:90ZU7} which is not available in
general case. Also, such a case-by-case parametrization has nothing to do with
the QR decomposition, so the work related to QR factorization is simply wasted.

Another approach, which is proposed by Kim and Vanderploeg, is to not
para\-me\-tri\-ze the positions for integration, and formulate the variables to
be integrated as $\mVec{y} = \begin{bmatrix} \mVec{q}^T, & \mVec{s}^T
\end{bmatrix}$. The authors then project the positions onto the constraints
surface orthogonally to some fixed hyperplane. We'll follow this approach, but
our projection will be orthogonal to the constraint surface.

Assuming, that $\mVec{q}^{(0)}$ is an inaccurate position provided by the
integrator (apart from constraints) we may search for the corrected position
$\mVec{q}$ by the following projection orthogonal to the constraints surface at
$\mVec{q}$
%%%%%%%%%%%%%%%%%%%%%%%%%%%%%%%%%%%%%%%%%%%%%%%%%%%%%%%%%%%%%%%%%%%%%%%%%%%%%
\begin{subequations}
\begin{align}
  (\mVec{q} - \mVec{q}^{(0)})^{T} (\mVec{q} - \mVec{q}^{(0)})
    & \rightarrow \min
  \\
  \text{w.r.t.}\;\;\mVec{\Phi}\mOf{\mVec{q}}
    & = \mVec{0}
\end{align}
\end{subequations}
%%%%%%%%%%%%%%%%%%%%%%%%%%%%%%%%%%%%%%%%%%%%%%%%%%%%%%%%%%%%%%%%%%%%%%%%%%%%%
This is equivalent to the following nonlinear problem
%%%%%%%%%%%%%%%%%%%%%%%%%%%%%%%%%%%%%%%%%%%%%%%%%%%%%%%%%%%%%%%%%%%%%%%%%%%%%
\begin{subequations}
\begin{align}
  \mVec{\Phi}\mOf{\mVec{q}} & = \mVec{0}
  \\
  \mMat{Q}_2^T\mOf{\mVec{q}}(\mVec{q} - \mVec{q}^{(0)}) &= \mVec{0}
\end{align}
\end{subequations}
%%%%%%%%%%%%%%%%%%%%%%%%%%%%%%%%%%%%%%%%%%%%%%%%%%%%%%%%%%%%%%%%%%%%%%%%%%%%%
which may be numerically solved by the following Newton iteration
%%%%%%%%%%%%%%%%%%%%%%%%%%%%%%%%%%%%%%%%%%%%%%%%%%%%%%%%%%%%%%%%%%%%%%%%%%%%%
\begin{subequations}
\begin{align}
  \begin{bmatrix}
    \mMat{\Phi}_{\mVec{q}}^{(k)} \\
    \mMat{J}_{\parallel}^{(k)}
  \end{bmatrix}
  \mVec{\Delta q}^{(k+1)}
  & = - \begin{bmatrix}
    \mVec{\Phi}^{(k)} \\
    \mMat{Q}_2^{T\,(k)} \left(\mVec{q}^{(k)} - \mVec{q}^{(0)}\right)
  \end{bmatrix}
  \\
  \mVec{q}^{(k+1)} &= \mVec{q}^{(k)} + \mVec{\Delta q}^{(k+1)}
\end{align}
\end{subequations}
%%%%%%%%%%%%%%%%%%%%%%%%%%%%%%%%%%%%%%%%%%%%%%%%%%%%%%%%%%%%%%%%%%%%%%%%%%%%%
where
%%%%%%%%%%%%%%%%%%%%%%%%%%%%%%%%%%%%%%%%%%%%%%%%%%%%%%%%%%%%%%%%%%%%%%%%%%%%%
\begin{equation}
  \mMat{J}_{\parallel}\mOf{\mVec{q}} = \frac{\partial}{\partial \mVec{q}}\left\{
    \mMat{Q}_2^{T}\mOf{\mVec{q}} \cdot \left(\mVec{q} - \mVec{q}^{(0)}\right)
  \right\}
\end{equation}
%%%%%%%%%%%%%%%%%%%%%%%%%%%%%%%%%%%%%%%%%%%%%%%%%%%%%%%%%%%%%%%%%%%%%%%%%%%%%
In our circle example we have
%%%%%%%%%%%%%%%%%%%%%%%%%%%%%%%%%%%%%%%%%%%%%%%%%%%%%%%%%%%%%%%%%%%%%%%%%%%%%
\begin{equation}
  \mMat{Q}_2^T \cdot \left(\mVec{q} - \mVec{q}^{(0)}\right)
  = \frac{q_1 q_2^{(0)} - q_2 q_1^{(0)}}{\sqrt{q_1^2 + q_2^2}}
  = \begin{bmatrix} q_2^{(0)}, & -q_1^{(0)} \end{bmatrix} \mMat{Q}_1
\end{equation}
%%%%%%%%%%%%%%%%%%%%%%%%%%%%%%%%%%%%%%%%%%%%%%%%%%%%%%%%%%%%%%%%%%%%%%%%%%%%%
and it may be shown, that
%%%%%%%%%%%%%%%%%%%%%%%%%%%%%%%%%%%%%%%%%%%%%%%%%%%%%%%%%%%%%%%%%%%%%%%%%%%%%
\begin{equation}
  \frac{\dif}{\dif t}\left\{
    \mMat{Q}_2^T \cdot \left(\mVec{q} - \mVec{q}^{(0)}\right)
  \right\}
  = \frac{\begin{bmatrix} q_2^{(0)}, & -q_1^{(0)} \end{bmatrix}}{\mVnorm{\mVec{q}}}
    \left(\mMat{I} + \mMat{Q}_1\mMat{Q}_1^T\right) \mVec{\dot q}
  = \mMat{J}_{\parallel} \mVec{\dot q}
\end{equation}
%%%%%%%%%%%%%%%%%%%%%%%%%%%%%%%%%%%%%%%%%%%%%%%%%%%%%%%%%%%%%%%%%%%%%%%%%%%%%
thus
%%%%%%%%%%%%%%%%%%%%%%%%%%%%%%%%%%%%%%%%%%%%%%%%%%%%%%%%%%%%%%%%%%%%%%%%%%%%%
\begin{equation}
  \mMat{J}_{\parallel}
= \frac{\begin{bmatrix} q_2^{(0)}, & -q_1^{(0)} \end{bmatrix}}{\mVnorm{\mVec{q}}}
  \left(\mMat{I} + \mMat{Q}_1\mMat{Q}_1^T\right)
\end{equation}

\section{Determining the ODEs (QR method)}

Our RHS function will compute the following vector of derivatives
%%%%%%%%%%%%%%%%%%%%%%%%%%%%%%%%%%%%%%%%%%%%%%%%%%%%%%%%%%%%%%%%%%%%%%%%%%%%%
\begin{equation}
  \mVec{\dot y} = \begin{bmatrix}
    \mVec{\dot q} \\
    \mVec{\dot s}
  \end{bmatrix}
\end{equation}
%%%%%%%%%%%%%%%%%%%%%%%%%%%%%%%%%%%%%%%%%%%%%%%%%%%%%%%%%%%%%%%%%%%%%%%%%%%%%
The $\mVec{\dot q}$ is already defined by \eqref{eq:38DY7}, and for ${\dot s}$
we use fact that $s = \mMat{Q}_2^T \mVec{\dot q}$, so
%%%%%%%%%%%%%%%%%%%%%%%%%%%%%%%%%%%%%%%%%%%%%%%%%%%%%%%%%%%%%%%%%%%%%%%%%%%%%
\begin{equation}
  {\dot s} =  \frac{\dif}{\dif t}\left\{\mMat{Q}_2^T\mVec{\dot q}\right\}
  = \mMat{\dot Q}_2^T \mVec{\dot q}
  + \mMat{Q}_2^T \mVec{\ddot q}
  = \mMat{\dot Q}_2^T \mVec{Q}_2 s
  + \mMat{Q}_2^T \mMat{Q}_1 \mMat{R}_1^{-T} \mVec{\Gamma}
  + \mMat{Q}_2^T \mMat{Q}_2 a
  = a
\end{equation}
%%%%%%%%%%%%%%%%%%%%%%%%%%%%%%%%%%%%%%%%%%%%%%%%%%%%%%%%%%%%%%%%%%%%%%%%%%%%%
because $\mMat{\dot Q}_2^T\mMat{Q}_2 = 0$ \footnote{Note, that
$\mMat{\dot Q}_2^T\mMat{Q}_2 = - \mMat{Q}_2^T \mMat{\dot Q}_2$ (because
$\mMat{Q}_2^T\mMat{Q}_2 = 1 = const)$. But in our (1-DOF) case it's it must
hold $\mMat{\dot Q}_2^T\mMat{Q}_2 = \mMat{Q}_2^T\mMat{\dot Q}_2$ at the same
time, so the product may only be zero.}.
%%%%%%%%%%%%%%%%%%%%%%%%%%%%%%%%%%%%%%%%%%%%%%%%%%%%%%%%%%%%%%%%%%%%%%%%%%%%%
The variable $a$ might be determined from the equation of motion
\eqref{eq:X83SF} and \eqref{eq:EKLKN} from which we simply get
%%%%%%%%%%%%%%%%%%%%%%%%%%%%%%%%%%%%%%%%%%%%%%%%%%%%%%%%%%%%%%%%%%%%%%%%%%%%%
\begin{equation}
  a = 0
\end{equation}
%%%%%%%%%%%%%%%%%%%%%%%%%%%%%%%%%%%%%%%%%%%%%%%%%%%%%%%%%%%%%%%%%%%%%%%%%%%%%
Finally, our ODEs may be written as
%%%%%%%%%%%%%%%%%%%%%%%%%%%%%%%%%%%%%%%%%%%%%%%%%%%%%%%%%%%%%%%%%%%%%%%%%%%%%
\begin{subequations}
  \begin{align}
    \mVec{\dot q} &= \mMat{Q}_2 s
    \\
    {\dot s} &= 0
  \end{align}
\end{subequations}
%%%%%%%%%%%%%%%%%%%%%%%%%%%%%%%%%%%%%%%%%%%%%%%%%%%%%%%%%%%%%%%%%%%%%%%%%%%%%

%%\section{Results (QR method)}
%%
%%In our experiments we run the integration for 1000 seconds and plot the
%%trajectory. We start from point $\mVec{q}_0 = \begin{bmatrix}1, &
%%0\end{bmatrix}^T$ and velocity $s = 1$. In an ideal situation the point should
%%move on the unit circle centered at zero.
%%
%%In the first experiment the equations of motion are integrated using ODE45 with
%%QR factorization and ``internal'' projections. By ``internal'' projections we
%%mean, that the projections are made only internally within the RHS procedure to
%%compute $\mMat{Q}_2(q)$, but the integrator still uses state $q^{(0)}$ which is
%%not projected. The resultant trajectory is plotted on
%%figure~\ref{fig:QEQGJ}.
%%%%%%%%%%%%%%%%%%%%%%%%%%%%%%%%%%%%%%%%%%%%%%%%%%%%%%%%%%%%%%%%%%%%%%%%%%%%%%%
%%\begin{figure}[htbp]
%%  \begin{center}
%%    \input{qrcirc1.tex}
%%    \caption{Trajectory of the point for ODE45 with QR and internal projection}
%%    \label{fig:QEQGJ}
%%  \end{center}
%%\end{figure}
%%%%%%%%%%%%%%%%%%%%%%%%%%%%%%%%%%%%%%%%%%%%%%%%%%%%%%%%%%%%%%%%%%%%%%%%%%%%%%%
%%
%%In the second experiment the equations of motion are integrated using ODE45 with
%%QR factorization and ``internal'' + ``external'' projections. We simply perform
%%few (up to five) integration steps and stop integration, then we project
%%position at the end and start the integration (for the next few steps) starting
%%from the updated position. The trajectory is shown on figure \ref{fig:S4ETL}.
%%%%%%%%%%%%%%%%%%%%%%%%%%%%%%%%%%%%%%%%%%%%%%%%%%%%%%%%%%%%%%%%%%%%%%%%%%%%%%%
%%\begin{figure}[htbp]
%%  \begin{center}
%%    \input{qrcirc2.tex}
%%    \caption{Trajectory of the point for ODE45 with QR and internal + external projection}
%%    \label{fig:S4ETL}
%%  \end{center}
%%\end{figure}
%%%%%%%%%%%%%%%%%%%%%%%%%%%%%%%%%%%%%%%%%%%%%%%%%%%%%%%%%%%%%%%%%%%%%%%%%%%%%%%
%%
%%\section{QR-update method}
%%
%%The ``QR-update'' method is the one proposed by Kim and Vanderploeg. The method
%%uses fixed coordinate system $\mMat{\bar Q}_1 = \mMat{Q}_1\mOf{\mVec{\bar q}}$,
%%$\mMat{\bar Q}_2 = \mVec{Q}_2\mOf{\mVec{\bar q}}$ generated at some point
%%$\mVec{\bar q}$ satisfying constraints and an updated coordinate system
%%$\mVec{\tilde Q}_1\mOf{\mVec{q}}$, $\mVec{\tilde Q}_2\mOf{\mVec{q}}$ at
%%$\mVec{q}$ in a neighborhood of $\mVec{\bar q}$.
%%
%%
%%In what follows, we'll express the solution to nonlinear constraints as
%%%%%%%%%%%%%%%%%%%%%%%%%%%%%%%%%%%%%%%%%%%%%%%%%%%%%%%%%%%%%%%%%%%%%%%%%%%%%%%
%%\begin{equation}
%%  \mVec{q} = \mMat{\bar Q}_1 w + \mMat{\bar Q}_2 z
%%\end{equation}
%%%%%%%%%%%%%%%%%%%%%%%%%%%%%%%%%%%%%%%%%%%%%%%%%%%%%%%%%%%%%%%%%%%%%%%%%%%%%%%
%%
%%\section{Solving the hidden constraints (QR-update method)}
%%
%%The hidden constraints \eqref{eq:IQT3I} are solved as
%%%%%%%%%%%%%%%%%%%%%%%%%%%%%%%%%%%%%%%%%%%%%%%%%%%%%%%%%%%%%%%%%%%%%%%%%%%%%%%
%%\begin{align}
%%  \mVec{\dot q} &= \mMat{\tilde Q}_2 {\dot z}
%%  \\
%%  \mVec{\ddot q} &= \mMat{\tilde Q}_2 {\ddot z} + \mMat{\tilde S} \Gamma
%%\end{align}
%%%%%%%%%%%%%%%%%%%%%%%%%%%%%%%%%%%%%%%%%%%%%%%%%%%%%%%%%%%%%%%%%%%%%%%%%%%%%%%
%%where $\mMat{\tilde Q}_2$ and $\mMat{\tilde S}$ are given by \eqref{eq:O1FSO}
%%and \eqref{eq:H3998} respectively.
%%
%%\section{Solving the nonlinear constraint (QR-update method)}
%%
%%To solve the non-linear constraints we formulate the following problem
%%%%%%%%%%%%%%%%%%%%%%%%%%%%%%%%%%%%%%%%%%%%%%%%%%%%%%%%%%%%%%%%%%%%%%%%%%%%%%%
%%\begin{subequations}
%%\begin{align}
%%  \mMat{\Phi}\mOf{\mVec{q}} &= 0
%%  \\
%%  \mMat{\bar Q}_2^T \left(\mVec{q} - \mVec{\bar q}\right) &= z
%%\end{align}
%%\end{subequations}
%%%%%%%%%%%%%%%%%%%%%%%%%%%%%%%%%%%%%%%%%%%%%%%%%%%%%%%%%%%%%%%%%%%%%%%%%%%%%%%
%%this is solved by the following Newton iteration
%%%%%%%%%%%%%%%%%%%%%%%%%%%%%%%%%%%%%%%%%%%%%%%%%%%%%%%%%%%%%%%%%%%%%%%%%%%%%%%
%%\begin{subequations}
%%\begin{align}
%%  \begin{bmatrix}
%%    \mMat{\Phi}_{\mVec{q}}^{(k)} \\
%%    \mMat{\bar Q}_2^T
%%  \end{bmatrix}
%%  \mVec{\Delta q}^{(k+1)}
%%& = - \begin{bmatrix}
%%  \mVec{\Phi}^{(k)} \\
%%  0
%%\end{bmatrix}
%%\\
%%\mVec{q}^{(k+1)}
%%&= \mVec{q}^{(k)} + \mVec{\Delta q}^{(k+1)}
%%\end{align}
%%\end{subequations}
%%%%%%%%%%%%%%%%%%%%%%%%%%%%%%%%%%%%%%%%%%%%%%%%%%%%%%%%%%%%%%%%%%%%%%%%%%%%%%%
%%with $\mVec{q}^{(0)} = \mVec{\bar q} + \mMat{\bar Q}_2 z$.
%%
%%\section{Determining the ODEs (QR-update method)}
%%
%%The independent accelerations ${\ddot z}$ are determined as
%%%%%%%%%%%%%%%%%%%%%%%%%%%%%%%%%%%%%%%%%%%%%%%%%%%%%%%%%%%%%%%%%%%%%%%%%%%%%%%
%%\begin{equation}
%%  {\ddot z} = - \left(\mMat{\tilde Q}_2^T\mMat{\tilde Q}_2\right)^{-1}
%%                \mMat{\tilde Q}_2^T \mMat{\tilde S} \Gamma
%%\end{equation}
%%%%%%%%%%%%%%%%%%%%%%%%%%%%%%%%%%%%%%%%%%%%%%%%%%%%%%%%%%%%%%%%%%%%%%%%%%%%%%%
%%and the resultant ODEs may be formulated by defining the following independent
%%variables
%%%%%%%%%%%%%%%%%%%%%%%%%%%%%%%%%%%%%%%%%%%%%%%%%%%%%%%%%%%%%%%%%%%%%%%%%%%%%%%
%%\begin{equation}
%%  \mVec{y} = \begin{bmatrix} z \\ {\dot z} \end{bmatrix}
%%\end{equation}
%%%%%%%%%%%%%%%%%%%%%%%%%%%%%%%%%%%%%%%%%%%%%%%%%%%%%%%%%%%%%%%%%%%%%%%%%%%%%%%
%%and then
%%%%%%%%%%%%%%%%%%%%%%%%%%%%%%%%%%%%%%%%%%%%%%%%%%%%%%%%%%%%%%%%%%%%%%%%%%%%%%%
%%\begin{equation}
%%  \mVec{\dot y} = \begin{bmatrix}
%%    y_2
%%    \\
%%    - \left(\mMat{\tilde Q}_2^T\mMat{\tilde Q}_2\right)^{-1}
%%      \mMat{\tilde Q}_2^T \mMat{\tilde S} \Gamma
%%  \end{bmatrix}
%%\end{equation}
%%%%%%%%%%%%%%%%%%%%%%%%%%%%%%%%%%%%%%%%%%%%%%%%%%%%%%%%%%%%%%%%%%%%%%%%%%%%%%%
%%
%%\section{Results (QR-update methods)}
%%
%%The trajectory from QR-update method is presented in figure \ref{fig:C1W1M}.
%%The error in the kinetic energy is presented on figures \ref{fig:VV69A} and
%%\ref{fig:5MXRG} for default ODE45 setting and for tolerances set to $10^{-9}$
%%respectively.
%%
%%%%%%%%%%%%%%%%%%%%%%%%%%%%%%%%%%%%%%%%%%%%%%%%%%%%%%%%%%%%%%%%%%%%%%%%%%%%%%%
%%\begin{figure}[htbp]
%%  \begin{center}
%%    \input{qrcirc3q.tex}
%%    \caption{Trajectory of the point for ODE45 with QR-update}
%%    \label{fig:C1W1M}
%%  \end{center}
%%\end{figure}
%%%%%%%%%%%%%%%%%%%%%%%%%%%%%%%%%%%%%%%%%%%%%%%%%%%%%%%%%%%%%%%%%%%%%%%%%%%%%%%
%%
%%%%%%%%%%%%%%%%%%%%%%%%%%%%%%%%%%%%%%%%%%%%%%%%%%%%%%%%%%%%%%%%%%%%%%%%%%%%%%%
%%\begin{figure}[htbp]
%%  \begin{center}
%%    \input{qrcirc3e.tex}
%%    \caption{Error in kinetic energy of the point for ODE45 with default
%%             tolerances and QR-update}
%%    \label{fig:VV69A}
%%  \end{center}
%%\end{figure}
%%%%%%%%%%%%%%%%%%%%%%%%%%%%%%%%%%%%%%%%%%%%%%%%%%%%%%%%%%%%%%%%%%%%%%%%%%%%%%%
%%
%%%%%%%%%%%%%%%%%%%%%%%%%%%%%%%%%%%%%%%%%%%%%%%%%%%%%%%%%%%%%%%%%%%%%%%%%%%%%%%
%%\begin{figure}[htbp]
%%  \begin{center}
%%    \input{qrcirc3e2.tex}
%%    \caption{Error in kinetic energy of the point for ODE45 with tolerances set
%%             to 1E-9 and QR-update}
%%    \label{fig:5MXRG}
%%  \end{center}
%%\end{figure}
%%%%%%%%%%%%%%%%%%%%%%%%%%%%%%%%%%%%%%%%%%%%%%%%%%%%%%%%%%%%%%%%%%%%%%%%%%%%%%%


% P. Tomulik: my bibliography
%%\bibliographystyle{spmpsci}
%%\bibliography{readme}
% P. Tomulik: end

\end{document}
% end of file template.tex

% \label{?:45K37}
% \label{?:GN2B2}
% \label{?:TCRQA}
% \label{?:X9Z2K}
% \label{?:2A6JG}
% \label{?:WBSNK}

% vim: set expandtab syntax=tex tabstop=2 shiftwidth=2:
% vim: set foldmethod=marker foldcolumn=4:
